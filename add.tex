\documentclass[a4paper, 10pt, twocolumn]{article}
 
%%%%%%%%%%%%%%%%%%%%%%%%%%%%%%%%%%%%%%%%%%%%%%%%%%%%%%%%%%%%%%%%%%%%%%%%%%%%%%
% additional packages 
%\usepackage[dvips]{graphicx}		% optional for graphics
\usepackage{graphicx}		% optional for graphics
\usepackage{tabularx}						% optional for tables
\usepackage{multirow}						% optional for tables
\usepackage{url}								% optional for internet links
\usepackage[ansinew]{inputenc}
\usepackage[small,bf]{caption} 	% for the captions below figures and tables
\usepackage{parskip}
\usepackage{titlesec}
\usepackage{amsmath}						% essential for equations
\usepackage{abstract}
\usepackage{xcolor}							%%bmr - Schriftfarbe
\usepackage{fancyhdr}						%%bmr - Kopf- und Fusszeilen
\usepackage{lipsum} \setlipsumdefault{1}
\usepackage[pdftex]{hyperref}		%%Dokumenteneigenschaften
\usepackage[english]{babel}			%% English- erweiterte Silbentrennung und �bersetzung von Begriffen
%\usepackage{cite} 							%% erlaubt Einbindung von bibrefs
\usepackage{ragged2e}						%% Flattersa
\usepackage[backend=bibtex,style=ACM-Reference-Format]{biblatex}
%%%%%%%%%%%%%%%%%%%%%%%%%%%%%%%%%%%%%%%%%%%%%%%%%%%%%%%%%%%%%%%%%%%%%%%%%%%%%%
\renewcommand{\rmdefault}{ptm}  	% Times New Roman
\newcommand{\citationneeded}[1][]{\textsuperscript{\color{black} [citation needed]}}

%\bibliographystyle{ACM-Reference-Format}
\bibliography{bibliography.bib}

%%Mit den folgenden Befehlen lassen sich die Dokumenteigenschaften des erstellten PDF Dokuments nach den eigenen W�nschen anpassen:
\hypersetup{pdfauthor = {H. Baumgartner, W. Hoeg, T. Sporer (Ver. 05/2018)},pdftitle = {Template (LaTeX format) for contributions for the ICSA},pdfsubject = {Verband Deutscher Tonmeister (VDT)},pdfkeywords = {keyword 1, keyword 2, etcetera},pdfcreator = {Anwendung},pdfproducer = {PDF erstellt mit}} 
%%Im Adobe Reader lassen diese sich dann mit STRG-D oder "Datei, Eigenschaften�" betrachten.

%%% Bemerkungen: (Unter Verwendung von Elementen einer DAGA-Vorlage)  (Using elements of a DAGA template)

%%% Anpassung fue reviewed Papers

\titleformat{\section}{\normalfont\Large\bfseries}{\thesection.}{11pt}{}
\titleformat{\subsection}{\normalfont\large\bfseries}{\thesubsection.}{2pt}{}
\titleformat{\subsubsection}{\normalfont\normalsize\bfseries}{\thesubsubsection.}{6pt}{}

\setlength{\baselineskip}{0pt}
\setlength{\belowcaptionskip}{-10pt}
\setlength{\parskip}{6pt}

\titlespacing{\section}{0pt}{6pt}{0pt}
\titlespacing{\subsection}{0pt}{3pt}{0pt}
\titlespacing{\subsubsection}{0pt}{0pt}{0pt}

\renewcommand{\abstractnamefont}{\normalfont\large\bfseries}
\renewcommand{\abstracttextfont}{\normalfont\normalsize}

\renewcommand{\thefigure}{\arabic{figure}}
\addto\captionsenglish{%
\renewcommand{\figurename}{Fig.}%
\renewcommand{\tablename}{Tab.}%
}
\renewcommand{\figurename}{Fig.}%
\renewcommand{\tablename}{Tab.}
% Strongly discourage hyphenation
\hyphenpenalty=5000
\tolerance=1000

%%%%%%%%%%%%%%%%%%%%%%%%%%%%%%%%%%%%%%%%%%%%%%%%%%%%%%%%%%%%%%%%%%%%%%%%%%%%%%
\newcolumntype{Y}{>{\centering\arraybackslash}X}	% For tables

%%%%%%%%%%%%%%%%%%%%%%%%%%%%%%%%%%%%%%%%%%%%%%%%%%%%%%%%%%%%%%%%%%%%%%%%%%%%%%
% Definitions page and text marges
\addtolength{\textwidth}{2.1cm}
\addtolength{\topmargin}{-2.7cm}
\addtolength{\oddsidemargin}{-1 cm}
\addtolength{\textheight}{3.8cm}
\setlength{\columnsep}{0.7cm}
\setlength{\voffset}{0.85cm}
\setlength{\headsep}{0.6cm}

%%%%%%%%%%%%%%%%%%%%%%%%%%%%%%%%%%%%%%%%%%%%%%%%%%%%%%%%%%%%%%%%%%%%%%%%%%%%%%
%\pagestyle{empty} 					% no page numbers etc.
\pagestyle{fancy} 
\urlstyle{rm}
%%%%%%%%%%%%%%%%%%%%%%%%%%%%%%%%%%%%%%%%%%%%%%%%%%%%%%%%%%%%%%%%%%%%%%%%%%%%%%
% START OF THE DOCUMENT
%%%%%%%%%%%%%%%%%%%%%%%%%%%%%%%%%%%%%%%%%%%%%%%%%%%%%%%%%%%%%%%%%%%%%%%%%%%%%%
\begin{document}
\fancyhf{} %alle Kopf- und Fusszeilenfelder bereinigen 

\chead{\underline{\sffamily\large\textcolor{gray}{5th International Conference on Spatial Audio ICSA, September 2019}}}
\renewcommand{\headrulewidth}{0pt}
\rhead{}
\renewcommand{\headrulewidth}{0pt}
\rhead{}

\fancyfoot[C]{\footnotesize $^*$ Please note that the papers at ICSA can be published by VDT, in print, online and as PDF download.}

%%%%%%%%%%%%%%%%%%%%%%%%%%%%%%%%%%%%%%%%%%%%%%%%%%%%%%%%%%%%%%%%%%%%%%%%%%%%%%
% FORMATION OF THE TITLE
%%%%%%%%%%%%%%%%%%%%%%%%%%%%%%%%%%%%%%%%%%%%%%%%%%%%%%%%%%%%%%%%%%%%%%%%%%%%%%
\date{}											% no date on page number 1

\title{
%%%spo
\vspace{-5mm}%
\mbox{\includegraphics[height=4cm]{VDT}}
%
\vspace{7mm}%
%
\begin{center}
\textbf{\Huge Ambisonics Decoder Description (.ADD)}\\
{ Presented  $^*$ by VDT.}
\end{center}
\vspace{10mm}%
 %~\newline
\begin{center}
\textbf\mdseries Developing a new file format for Ambisonics decoding matrices
\end{center}
%
\mbox{}\vspace{-11mm}
%
}

%%%%%%%%%%%%%%%%%%%%%%%%%%%%%%%%%%%%%%%%%%%%%%%%%%%%%%%%%%%%%%%%%%%%%%%%%%%%%%
% Names and affiliation of authors
\author{ %
%
G. Arlauskas, J. Ohland, H. Schaar\\
%
\textit{\large %
University of Applied Sciences Darmstadt, Germany, Email: jonas.ohland@stud.h-da.de
}\\
%
}
%

\twocolumn[
\maketitle
\thispagestyle{fancy}
%\thispagestyle{empty} 			% no page numbers etc.
%
%%%%%%%%%%%%%%%%%%%%%%%%%%%%%%%%%%%%%%%%%%%%%%%%%%%%%%%%%%%%%%%%%%%%%%%%%%%%%%
% Start of the manuscript
%%%%%%%%%%%%%%%%%%%%%%%%%%%%%%%%%%%%%%%%%%%%%%%%%%%%%%%%%%%%%%%%%%%%%%%%%%%%%%

{\vspace{-3mm}}
\begin{onecolabstract}
{\vspace{3mm}}

Different software solutions have been developed for the calculation and implementation of Ambisonics decoding matrices. The present paper presents and describes a new data file format which can be used as an intermediate between solutions.

Currently available software solutions use particular data conventions causing difficult compatibility and exchangeability. In the present work an open-source toolkit is developed for storing, handling and using Ambisonics decoding matrices. The toolkit includes tools for conversion from common matrix data conventions to the ADD-format and back, calculating decoding matrices, decoding Ambisonics signals and extracting existing matrices from external decoding tools. 

The new ADD-format and toolkit enables increased flexibility in production workflows and eliminates the drawbacks and limitations regarding compatibility between software solutions.

\end{onecolabstract}
{\vspace{8mm}}]

\section{Introduction} \label{sec:introduction}

\subsection{Ambisonics}

Ambisonics is a 3D audio representation approach based on spherical harmonics. Compared to traditional multichannel audio where each channel contains the signal for one loudspeaker, in Ambisonics the channels contain information about certain properties of the acoustic field.

It was first developed in the 1970s by the British National Research Development Corporation for Broadcasting purposes.\citationneeded Recently the format has found new popularity as higher order implementations in virtual reality applications and special mutlichannel setups.



%%%%%%%%%%%%%%%%%%%%%%%%%%%%%%%%%%%%%%%%%%%%%%%%%%%%%%%%%%%%%%%%%%%%%%%%%%%%%%
\section{Ambisonics Encoding/Decoding} \label{sec:AmbisonicsEncodingDecoding}

\subsection{Normalization}

For succesful reconstruction of the sound field encoding and decoding have to agree on the method by which the spherical harmonic compononents are normalized. There are different approaches to calculating the normalization factor $N$ of ambisonics channel $m$ with order $\ell$ . In the following the most common ones are described.

\subsubsection{SN3D}

With SN3D or Schmidt semi-normalisation no component will ever exceed the peak vlaue of the 0th order component for single point sources. \autocite{ambix} Since the proposal of the AmbiX format it has been widely adopted in modern ambisonics software development.

\[
    N_{\ell,m}^{SN3D} = \sqrt{(2-\delta_m)\cfrac{\ell-|m|!}{\ell+|m|!}},\delta_m 
    \begin{cases} 
    1 & \quad \text{if m = 0}\\
    1 & \quad \text{if m $\neq$ 0}
    
    \end{cases}
\]

\subsubsection{N3D}

N3D differs from SN3D only by a scaling factor. It esnures qual power of the encoded components in the case of a perfectly diffuse 3D field.\autocite{daniel} It is used in MPEG-H.

\[
    N_{\ell, m}^{\mathrm{N} 3 \mathrm{D}}=N_{\ell, m}^{\mathrm{SN} 3 \mathrm{D}} \sqrt{2 \ell+1}    
\]

(For further information on normalization see \autocite{norm})

%%%%%%%%%%%%%%%%%%%%%%%%%%%%%%%%%%%%%%%%%%%%%%%%%%%%%%%%%%%%%%%%%%%%%%%%%%%%%%
\section{Motivation} \label{sec:Motivation}

Lorem Ipsum \autocite{daniel}

%%%%%%%%%%%%%%%%%%%%%%%%%%%%%%%%%%%%%%%%%%%%%%%%%%%%%%%%%%%%%%%%%%%%%%%%%%%%%%
\section{Implementation} \label{sec:Implementation}

Lorem Ipsum
%%%%%%%%%%%%%%%%%%%%%%%%%%%%%%%%%%%%%%%%%%%%%%%%%%%%%%%%%%%%%%%%%%%%%%%%%%%%%%


\section{Conclusion }
The final manuscript should be submitted in electronic form to the editors of the proceedings prior to the conference. In exceptional circumstances a later date must be agreed by the editors. If you have any questions, please ask the program team of ICSA (icsa-proceedings@imt.rz.tu-ilmenau.de).

Authors must make sure that they hold the copyrights of all of content in their manuscript. This is particularly important to any kind of picture, drawing etc.

Authors should not include any company logo or advertising information of a real product in the manuscript; otherwise the paper will be rejected from the Proceedings and handled like a Product presentation. 

%%%%%%%%%%%%%%%%%%%%%%%%%%%%%%%%%%%%%%%%%%%%%%%%%%%%%%%%%%%%%%%%%%%%%%%%%%%%%%
% REFERENCES
%%%%%%%%%%%%%%%%%%%%%%%%%%%%%%%%%%%%%%%%%%%%%%%%%%%%%%%%%%%%%%%%%%%%%%%%%%%%%%
\renewcommand\refname{}
\section{References}

\begingroup
\RaggedRight 		%% Flattersatz

\printbibliography
\endgroup
%%%%%%%%%%%%%%%%%%%%%%%%%%%%%%%%%%%%%%%%%%%%%%%%%%%%%%%%%%%%%%%%%%%%%%%%%%%%%%
\end{document}

%
%Beispiel bibDatei:
%@inproceedings
%{
%Derntl2004Pattern,author= {Derntl, Michael and MotschnigPitrik, Renate},
%title = {Patterns for blended, PersonCentered learning: strategy, concepts,experiences, and evaluation},
%booktitle = {SAC '04: Proceedings of the 2004 ACM symposium on Applied computing},
%year = {2004},
%isbn = {1581138121},
%pages = {916923},
%location = {Nicosia, Cyprus},
%doi = {
%http://doi.acm.org/10.1145/967900.968087
%}

